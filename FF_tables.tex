\documentclass[8pt,]{article}
\usepackage{lmodern}
\usepackage{amssymb,amsmath}
\usepackage{ifxetex,ifluatex}
\usepackage{fixltx2e} % provides \textsubscript
\ifnum 0\ifxetex 1\fi\ifluatex 1\fi=0 % if pdftex
  \usepackage[T1]{fontenc}
  \usepackage[utf8]{inputenc}
\else % if luatex or xelatex
  \ifxetex
    \usepackage{mathspec}
  \else
    \usepackage{fontspec}
  \fi
  \defaultfontfeatures{Ligatures=TeX,Scale=MatchLowercase}
\fi
% use upquote if available, for straight quotes in verbatim environments
\IfFileExists{upquote.sty}{\usepackage{upquote}}{}
% use microtype if available
\IfFileExists{microtype.sty}{%
\usepackage{microtype}
\UseMicrotypeSet[protrusion]{basicmath} % disable protrusion for tt fonts
}{}
\usepackage[margin=1in]{geometry}
\usepackage{hyperref}
\hypersetup{unicode=true,
            pdfborder={0 0 0},
            breaklinks=true}
\urlstyle{same}  % don't use monospace font for urls
\usepackage{longtable,booktabs}
\usepackage{graphicx,grffile}
\makeatletter
\def\maxwidth{\ifdim\Gin@nat@width>\linewidth\linewidth\else\Gin@nat@width\fi}
\def\maxheight{\ifdim\Gin@nat@height>\textheight\textheight\else\Gin@nat@height\fi}
\makeatother
% Scale images if necessary, so that they will not overflow the page
% margins by default, and it is still possible to overwrite the defaults
% using explicit options in \includegraphics[width, height, ...]{}
\setkeys{Gin}{width=\maxwidth,height=\maxheight,keepaspectratio}
\IfFileExists{parskip.sty}{%
\usepackage{parskip}
}{% else
\setlength{\parindent}{0pt}
\setlength{\parskip}{6pt plus 2pt minus 1pt}
}
\setlength{\emergencystretch}{3em}  % prevent overfull lines
\providecommand{\tightlist}{%
  \setlength{\itemsep}{0pt}\setlength{\parskip}{0pt}}
\setcounter{secnumdepth}{0}
% Redefines (sub)paragraphs to behave more like sections
\ifx\paragraph\undefined\else
\let\oldparagraph\paragraph
\renewcommand{\paragraph}[1]{\oldparagraph{#1}\mbox{}}
\fi
\ifx\subparagraph\undefined\else
\let\oldsubparagraph\subparagraph
\renewcommand{\subparagraph}[1]{\oldsubparagraph{#1}\mbox{}}
\fi

%%% Use protect on footnotes to avoid problems with footnotes in titles
\let\rmarkdownfootnote\footnote%
\def\footnote{\protect\rmarkdownfootnote}

%%% Change title format to be more compact
\usepackage{titling}

% Create subtitle command for use in maketitle
\newcommand{\subtitle}[1]{
  \posttitle{
    \begin{center}\large#1\end{center}
    }
}

\setlength{\droptitle}{-2em}

  \title{}
    \pretitle{\vspace{\droptitle}}
  \posttitle{}
    \author{}
    \preauthor{}\postauthor{}
    \date{}
    \predate{}\postdate{}
  
\usepackage{longtable}\setlength{\LTleft}{0em}

\begin{document}

\textbf{7. Tables}

Table 1. Variables Used in PERMANOVA Models.

\begin{longtable}[]{@{}lrr@{}}
\toprule
Variable & Abbreviation & Source\tabularnewline
\midrule
\endhead
\textbf{Fire} & &\tabularnewline
Time Since Fire & TSF & MTBS\tabularnewline
Fire Frequency & FF & MTBS\tabularnewline
& &\tabularnewline
\textbf{Climate} & &\tabularnewline
Maximum Vapor Pressure Deficit & &\tabularnewline
\emph{Year of Fire} & vpdmax\_during & GRIDMET\tabularnewline
\emph{Year before Fire} & vpdmax\_before & GRIDMET\tabularnewline
\emph{Year after Fire} & vpdmax\_after & GRIDMET\tabularnewline
Maximum Temperature & &\tabularnewline
\emph{Year of Fire} & tmax\_during & GRIDMET\tabularnewline
\emph{Year before Fire} & tmax\_pre & GRIDMET\tabularnewline
\emph{Year after Fire} & tmax\_after & GRIDMET\tabularnewline
Precipitation & &\tabularnewline
\emph{Nov - May; 2 Years Before Fire} & ppt\_2pre &
GRIDMET\tabularnewline
\emph{Nov - May; 1 Year Before Fire} & ppt\_1pre &
GRIDMET\tabularnewline
\emph{Nov - May; After Fire} & ppt\_post & GRIDMET\tabularnewline
& &\tabularnewline
\textbf{Other} & &\tabularnewline
Folded Aspect & & Field Measurements\tabularnewline
Slope & & Field Measurements\tabularnewline
Elevation & & USGS\tabularnewline
Animal Unit Months per Hectare & AUM\_ha & BLM\tabularnewline
\bottomrule
\end{longtable}

\clearpage
\newpage

\begin{table}[!htbp] \centering 
  \caption{Results of linear mixed models testing the relationship between diversity indexes and cheatgrass abundance, while accounting for elevation. Study block was the random effect. Partial coefficient of determination was calculated from Jeager et al. (2016)} 
  \label{} 
\begin{tabular}{@{\extracolsep{5pt}}lcccc} 
\\[-1.8ex]\hline 
\hline \\[-1.8ex] 
 & \multicolumn{4}{c}{\textit{Dependent variable:}} \\ 
\cline{2-5} 
\\[-1.8ex] & Shannon-Weaver & Pielou Evenness & Number of Species & Beta Diversity \\ 
\hline \\[-1.8ex] 
 Cheatgrass Cover & $-$0.017$^{***}$ & $-$0.008$^{***}$ & $-$0.071$^{***}$ & 0.002 \\ 
  & (0.003) & (0.002) & (0.027) & (0.001) \\ 
  & & & & \\ 
 Elevation & 0.189$^{***}$ & 0.046 & 1.700$^{***}$ & 0.065$^{***}$ \\ 
  & (0.062) & (0.029) & (0.642) & (0.021) \\ 
  & & & & \\ 
 Constant & 1.214$^{***}$ & 0.652$^{***}$ & 8.538$^{***}$ & 0.288$^{***}$ \\ 
  & (0.103) & (0.052) & (1.121) & (0.037) \\ 
  & & & & \\ 
\hline \\[-1.8ex] 
partial R\textsuperscript{2}, Cheatgrass Cover & 0.65 & 0.51 & 0.24 & 0.08 \\
\hline 
\hline \\[-1.8ex] 
\textit{Note:}  & \multicolumn{4}{r}{$^{*}$p$<$0.1; $^{**}$p$<$0.05; $^{***}$p$<$0.01} \\ 
\end{tabular} 
\end{table}

\clearpage
\newpage

Table 3. PERMANOVA results for fire history and environmental factors
influencing post-fire \textbf{community composition}.

\begin{longtable}[]{@{}lrrrrrr@{}}
\toprule
& Df & SumsOfSqs & MeanSqs & F.Model & R2 &
Pr(\textgreater{}F)\tabularnewline
\midrule
\endhead
TSF & 1 & 0.1457 & 0.1457 & 2.0189 & 0.0713 & 0.0082\tabularnewline
FF & 1 & 0.2342 & 0.2342 & 3.2448 & 0.1146 & 0.0485\tabularnewline
vpdmax\_during & 1 & 0.2243 & 0.2243 & 3.1081 & 0.1098 &
0.0016\tabularnewline
tmax\_during & 1 & 0.0952 & 0.0952 & 1.3186 & 0.0466 &
0.1358\tabularnewline
tmax\_pre & 1 & 0.0776 & 0.0776 & 1.0757 & 0.0380 &
0.1790\tabularnewline
AUM\_ha & 1 & 0.2038 & 0.2038 & 2.8232 & 0.0997 & 0.2111\tabularnewline
TSF:FF & 1 & 0.1246 & 0.1246 & 1.7262 & 0.0610 & 0.0389\tabularnewline
Residuals & 13 & 0.9383 & 0.0722 & & 0.4591 &\tabularnewline
Total & 20 & 2.0436 & & & 1.0000 &\tabularnewline
\bottomrule
\end{longtable}

\clearpage
\newpage

Table 4. PERMANOVA results for fire history and environmental factors
influencing post-fire \textbf{beta diversity} (Whittaker's index).

\begin{longtable}[]{@{}lrrrrrr@{}}
\toprule
& Df & SumsOfSqs & MeanSqs & F.Model & R2 &
Pr(\textgreater{}F)\tabularnewline
\midrule
\endhead
FF & 1 & 0.2110 & 0.2110 & 2.4408 & 0.0783 & 0.0070\tabularnewline
ppt\_1pre & 1 & 0.5236 & 0.5236 & 6.0581 & 0.1943 &
0.0014\tabularnewline
tmax\_after & 1 & 0.1075 & 0.1075 & 1.2438 & 0.0399 &
0.0646\tabularnewline
ppt\_2pre & 1 & 0.3993 & 0.3993 & 4.6199 & 0.1482 &
0.0134\tabularnewline
TSF & 1 & 0.1450 & 0.1450 & 1.6779 & 0.0538 & 0.2493\tabularnewline
folded\_aspect & 1 & 0.1105 & 0.1105 & 1.2780 & 0.0410 &
0.4735\tabularnewline
Elevation & 1 & 0.0442 & 0.0442 & 0.5108 & 0.0164 &
0.9526\tabularnewline
ppt\_1pre:tmax\_after & 1 & 0.1162 & 0.1162 & 1.3447 & 0.0431 &
0.0255\tabularnewline
Residuals & 12 & 1.0371 & 0.0864 & & 0.3849 &\tabularnewline
Total & 20 & 2.6943 & & & 1.0000 &\tabularnewline
\bottomrule
\end{longtable}


\end{document}
