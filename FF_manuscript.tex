\documentclass[12pt,]{article}
\usepackage{lmodern}
\usepackage{amssymb,amsmath}
\usepackage{ifxetex,ifluatex}
\usepackage{fixltx2e} % provides \textsubscript
\ifnum 0\ifxetex 1\fi\ifluatex 1\fi=0 % if pdftex
  \usepackage[T1]{fontenc}
  \usepackage[utf8]{inputenc}
\else % if luatex or xelatex
  \ifxetex
    \usepackage{mathspec}
  \else
    \usepackage{fontspec}
  \fi
  \defaultfontfeatures{Ligatures=TeX,Scale=MatchLowercase}
\fi
% use upquote if available, for straight quotes in verbatim environments
\IfFileExists{upquote.sty}{\usepackage{upquote}}{}
% use microtype if available
\IfFileExists{microtype.sty}{%
\usepackage{microtype}
\UseMicrotypeSet[protrusion]{basicmath} % disable protrusion for tt fonts
}{}
\usepackage[margin=1in]{geometry}
\usepackage{hyperref}
\hypersetup{unicode=true,
            pdfborder={0 0 0},
            breaklinks=true}
\urlstyle{same}  % don't use monospace font for urls
\usepackage{graphicx,grffile}
\makeatletter
\def\maxwidth{\ifdim\Gin@nat@width>\linewidth\linewidth\else\Gin@nat@width\fi}
\def\maxheight{\ifdim\Gin@nat@height>\textheight\textheight\else\Gin@nat@height\fi}
\makeatother
% Scale images if necessary, so that they will not overflow the page
% margins by default, and it is still possible to overwrite the defaults
% using explicit options in \includegraphics[width, height, ...]{}
\setkeys{Gin}{width=\maxwidth,height=\maxheight,keepaspectratio}
\IfFileExists{parskip.sty}{%
\usepackage{parskip}
}{% else
\setlength{\parindent}{0pt}
\setlength{\parskip}{6pt plus 2pt minus 1pt}
}
\setlength{\emergencystretch}{3em}  % prevent overfull lines
\providecommand{\tightlist}{%
  \setlength{\itemsep}{0pt}\setlength{\parskip}{0pt}}
\setcounter{secnumdepth}{0}
% Redefines (sub)paragraphs to behave more like sections
\ifx\paragraph\undefined\else
\let\oldparagraph\paragraph
\renewcommand{\paragraph}[1]{\oldparagraph{#1}\mbox{}}
\fi
\ifx\subparagraph\undefined\else
\let\oldsubparagraph\subparagraph
\renewcommand{\subparagraph}[1]{\oldsubparagraph{#1}\mbox{}}
\fi

%%% Use protect on footnotes to avoid problems with footnotes in titles
\let\rmarkdownfootnote\footnote%
\def\footnote{\protect\rmarkdownfootnote}

%%% Change title format to be more compact
\usepackage{titling}

% Create subtitle command for use in maketitle
\newcommand{\subtitle}[1]{
  \posttitle{
    \begin{center}\large#1\end{center}
    }
}

\setlength{\droptitle}{-2em}

  \title{}
    \pretitle{\vspace{\droptitle}}
  \posttitle{}
    \author{}
    \preauthor{}\postauthor{}
    \date{}
    \predate{}\postdate{}
  
\usepackage{setspace}\doublespacing
\usepackage[left]{lineno}

\begin{document}

Running Head: Reburns Reduce Sagebrush Diversity.

Title: Repeated Fires Reduce Plant Diversity in Low-Elevation Wyoming
Big Sagebrush Ecosystems (1984-2014).

\vspace{5mm}

Authors:\newline
\newline
Adam L. Mahood *\newline
Department of Geography\newline
University of Colorado Boulder\newline
GUGG 110, 260 UCB\newline
Boulder, CO 80309-0260\newline
\newline
Jennifer K. Balch\newline
Department of Geography\newline
University of Colorado Boulder\newline
GUGG 110, 260 UCB \newline
Boulder, CO 80309-0260\newline
\newline
*Corresponding author\newline

\newpage

\linenumbers

\textbf{Abstract}

Sagebrush is one of the most imperiled ecosystems in Western North
America, having lost about half of its original 62 million hectare
extent. Annual grass invasions are known to be increasing wildfire
occurrence and burned area, but the lasting effects (\textgreater{} five
years post-fire) that the resulting reburns have on these plant
communities are unclear. We created a fire history atlas from 31 years
(1984 to 2014) of Landsat-derived fire data to sample along a fire
frequency gradient (zero to three fires) in an area of northern Nevada
that has experienced frequent fire in this time period. 32\% of our
study area (13,000 km\(^2\)) burned in large fires (over 404 ha) at
least once, 7\% burned twice, and 2\% burned three or more times. We
collected plant abundance data at 28 plots (N=7 per fire frequency),
with an average time-since-fire of 17 years. We examined fire's effect
on plant diversity using species accumulation curves, alpha-diversity
(Shannon's dominance, Pielou's evenness and number of species), and beta
diversity (Whittaker, Simpson, and Z indexes). For composition, we used
non-metric multidimensional scaling. We then used PERMANOVA models to
examine how disturbance history, temperature, precipitation and aridity
around the time of the fire affected subsequent community composition
and diversity.

One fire fundamentally changed community composition and reduced species
richness, and each subsequent fire reduced richness further. Alpha
diversity decreased after one fire. Beta diversity declined after the
third fire. Cover of exotics was 10\% higher in all burned plots, and
native cover was 20\% lower than in unburned plots, regardless of
frequency. PERMANOVA models showed fire frequency and antecedent
precipitation as the strongest predictors of beta diversity, while time
since fire and vapor pressure deficit for the year of the fire were the
strongest predictors of community composition. Given that a single fire
has such a marked effect on species composition, and repeated fires
reduce richness and beta diversity, we suggest that in lower elevation
big sagebrush systems fire should be minimized as much as possible,
perhaps even prescribed fire. Restoration efforts should be focused on
timing with wet years on cooler, wetter sites.

\emph{Key Words: Bromus tectorum, cheatgrass, fire, sagebrush, fire
frequency, repeated fire, biodiversity, community composition, Artemisia
tridentata ssp. wyomingensis}

\textbf{Introduction}

Wildfire activity has been increasing across the western U.S. since the
1980s (Westerling et al. 2006, Dennison et al. 2014, Westerling 2016,
Balch et al. 2017), and this is leading to concern among land managers
in the U.S. Great Basin (Miller et al. 2013, Integrated Rangeland Fire
Management Strategy Actionable Science Plan Team 2016, Chambers et al.
2017). This trend will likely continue as rising temperatures and more
frequent drought events increase the probability of fire (Krawchuk et
al. 2009, Moritz et al. 2012, Liu et al. 2013), and as these climatic
factors combine with increased human ignition pressure (Balch et al.
2017) and land use change (Bowman et al. 2011) to increase the length of
the fire season (Wotton and Flannigan 1993, Jolly et al. 2015). This
increased fire activity is one contributing factor to the loss of
approximately half of the area of sagebrush (\emph{Artemisia
tridentata}) shrubland communities, which once occupied over 600,000
km\(^2\) in the western US. Much of this land is now dominated by
cheatgrass (\emph{Bromus tectorum}) (Bradley and Mustard 2008), an
introduced annual grass (Davies 2011). This in turn is initiating a
positive feedback, wherein invading plants increase the probability of
fire, and increased fire activity stimulates more annual grass invasion
(D'Antonio and Vitousek 1992, Brooks et al. 2004, Balch et al. 2013).
The result is a fire return interval that has decreased from a
historical range of 100-342 years for intact sagebrush (Baker 2006,
Bukowski and Baker 2013) to 78 years in invaded areas (Balch et al.
2013), to as low as 3-5 years in cheatgrass-dominated areas in the Snake
River Plain (Whisenant 1990). This increase in fire activity results in
more areas that are burned multiple times, and the lasting effect this
has on plant communities' biodiversity and composition is relatively
unknown. There are relatively few studies on the impacts of fire after
more than 5 years (but see Beck et al. 2009, Reed-Dustin et al. 2016),
and fewer still that analyze the impacts of repeated fires in the same
location (Miller et al. 2013).

There are at least 40 vertebrate species of conservation concern
associated with sagebrush habitats (Rowland et al. 2006), including the
greater sage grouse (\emph{Centrocercus urophasianus}). Greater sage
grouse depends on sagebrush for its habitat and has been a management
priority by land managers (Chambers et al. 2017). Optimal shrub cover
for sage grouse is 15-25\% with over 15\% bunchgrasses and forbs (Beck
et al. 2009). Fire is one of the top 2 threats to the greater sage
grouse in the western part of its range (Brooks et al. 2015), and the
loss of sagebrush due to wildfire has contributed strongly to its
population declines over the past 30 years (Coates et al. 2016). Land
management agencies have linked fire management with long-term
conservation goals focused on sagebrush ecosystems and the greater sage
grouse (Chambers et al. 2017).

There is emerging consensus among researchers and land managers that
lower elevation Wyoming big sagebrush (\emph{A. tridentata} ssp.
\emph{wyomingensis}) ecosystems are not resilient to fire (Chambers et
al. 2014) and should be prevented from burning whenever possible, while
higher elevation Mountain big sagebrush (\emph{A. tridentata} ssp.
\emph{vaseyana}) ecosystems may still recover naturally (Hanna and
Fulgham 2015) or with restoration by seeding (Knutson et al. 2014).
Several authors have recommended attempting to reduce the size and
frequency of wildfire, and stopping the use of prescribed fire
(Whisenant 1990, Baker 2006, Lesica et al. 2007, Beck et al. 2009),
while also reducing grazing (Shinneman and Baker 2009, Ellsworth and
Kauffman 2013). Others have urged caution with the use of prescribed
fire (Davies et al. 2009, Reed-Dustin et al. 2016, Shinneman and McIlroy
2016). There has been disagreement in the past about the historical fire
return interval for Wyoming big sagebrush. It has been characterized as
being every 35 -- 100 years (Schmidt et al. 2002), every 100 -- 240
years (Baker 2006), to every 171 -- 342 years (Bukowski and Baker 2013).
This discrepancy has important management implications, leading to
disagreement as to which stressors or disturbances (e.g.~grazing, fire)
need to be increased or decreased in order to manage for healthy
sagebrush ecosystems. The lower estimations imply the system is
fire-dependent and requires frequent burning in order to persist, while
the upper estimates suggest fire-sensitivity.

Wyoming big sagebrush assemblages are generally agreed to be an
endangered ecosystem and fire and the invasive plants that generally
colonize afterwards are thought to be two major drivers of declining
biodiversity in this system (Davies et al. 2011). Cover of introduced
annual grass species have been mostly observed to be negatively related
to species richness and native diversity (Davies 2011, Gasch et al.
2013, Bansal and Sheley 2016), but over a 45 year period Anderson and
Inouye (2001) found that while introduced annual grass cover was
negatively correlated with cover of native species, species richness was
unrelated. While fire is strongly correlated with annual grass cover in
this system at regional scales (Balch et al. 2013), it has also been
shown to be an unimportant predictor variable for both exotic cover and
species richness in eastern Washington (Mitchell et al. 2016).

Post-fire communities of introduced annual grasses are affected by both
fire frequency and time since fire. Cheatgrass cover can increase
initially after fire, then stabilize above its pre-fire cover after 2-5
years (Reed-Dustin et al. 2016), but positive linear relationships
between time since fire and cheatgrass cover have also been observed
(Shinneman and Baker 2009), as well as areas where cheatgrass declined
and was replaced by perennial grasses (West and Yorks 2002, Hanna and
Fulgham 2015). Pre-fire community composition might explain the
inconsistency in results. Cheatgrass can come to dominate areas with
fire-intolerant natives post-fire, but in areas with pre-fire
populations of fire-tolerant species (e.g. \emph{Poa secunda}) these
species can regenerate following fire (Davies et al. 2012, but see
Bagchi et al. 2013).

Precipitation, temperature and aridity affect both fire occurrence and
the subsequent recovery of plant communities. Unlike most forested
systems in the western US, burned area in Great Basin sagebrush systems
is best predicted by antecedent precipitation (Abatzoglou and Kolden
2013, Pilliod et al. 2017). Precipitation also drives the invasion of
cheatgrass into lower elevation sagebrush systems (Chambers et al.
2007), which increases the probability of fire for several years due to
the persistence of the litter it leaves behind (Pilliod et al. 2017).
Cheatgrass invasion increases the continuity of fuels (Davies and Nafus
2013) and burned area (Balch et al. 2013), thereby reducing the number
of unburned patches that provide the native seed sources critical for
recolonizing burned areas. Unburned patches are essential for sagebrush
regeneration as almost every species in this genus is a seed obligate
and the seeds generally fall no more than 30 meters from the mother
plant (Meyer 1994). Once established, a sagebrush seedling needs to be
able to withstand drought conditions in the summer to survive and be
recruited into the population (Meyer 1994).

Here, we explored how sagebrush community composition and diversity
responded to increasing fire disturbance by constructing a fire history
atlas and sampling plant communities that burned zero to three times
between 1984 and 2014 in the Central Basin and Range ecoregion. We
constrained soil, ecological site type, elevation and climate and
sampled blocks of plots stratified along a gradient of 0-3 fires. Our
first hypothesis was that community composition would change drastically
between unburned and burned plots, but remain similar between burned
plots of different fire frequencies. This was our expectation because in
the Great Basin there are vast areas of sagebrush which are generally
unburned in the last 30+ years, and burned areas are almost always
completely dominated by cheatgrass, along with a handful of exotic forb
species and a native grass, \emph{Poa secunda}. These
cheatgrass-dominated areas all appear very similar, regardless of fire
frequency (Figure 1). But we suspected that there would be a signal on
plant diversity after multiple fires, due to selective pressure against
fire-intolerant plants. Thus our second hypothesis was that alpha
diversity (Shannon-Weaver index, Pielou Evenness, and the number of
species in a sampling unit), beta diversity (continuity or turnover of
species between plots) and the extrapolated species richness (with plots
pooled by fire frequency), would decrease with increasing fire
frequency. Our third hypothesis was that cheatgrass abundance would have
a negative relationship with plant diversity. Our fourth hypothesis was
that temperature, vapor pressure deficit and precipitation around the
time of the fire would exert a lasting influence over post-fire
community composition and diversity. This is based on evidence that the
effects of introducing species at the beginning of secondary succession
can be long-lasting (Veen et al. 2018), and in this system the
assemblage of species that are able to successfully colonize an area
after a fire depends on their abilities to compete for moisture and
tolerate drought (Meyer 1994).

\textbf{Methods}

\emph{Study area}

We conducted the study in a 13,000 square kilometer region in northern
Nevada (Figure 2). The region has hot, dry summers and cold, wet
winters. Annual precipitation averages 293 mm, falling mostly from
November to May. Mean temperatures range from 21.8 degrees Celsius in
July to -1.4 degrees Celsius in December (PRISM Climate Group 2016). The
region consists of mountain ranges that run north-south, and the
sagebrush ecosystems generally lie on the lower slopes of the mountains;
our sites ranged from 1272 to 1696 meters in elevation (median 1458, SD
99). From 1984 to 2014, 32\% (4096 square kilometers) of the study area
burned in large fires (over 500 acres) at least once, 7\% burned twice,
and 2\% burned three or more times (Table S1).

\emph{Site selection}

We used a block sampling design, with each block containing one site
from each of four fire frequencies (zero to three), and all fires
occurring at least five years before the study. We used geospatial data
representing ecosystem state factors (\emph{sensu} Amundson and Jenny
1997) to design a sampling scheme that constrained all other factors. We
used the LANDFIRE (Rollins 2009) biophysical setting layer to eliminate
all vegetation types except big sagebrush shrubland. The LANDFIRE data
has 62-68\% classification accuracy for shrublands (Zhu et al. 2006). We
used soil data from the Natural Resource Conservation Service to include
only areas in the Loamy 8-10 precipitation zone (Soil Survey Staff,
Natural Resources Conservation Service, United States Department of
Agriculture (USDA) 2016). We chose this particular zone simply because
it was the most common type in our study area within the big sagebrush
shrubland biophysical setting. We used the Land Treatment Digital
Library to exclude areas that had undergone intensive restoration
activities (Pilliod and Welty 2013). Excluding private and military
land, and areas more than five miles from a road eliminated impractical
plot locations and held human influence somewhat constant.

We accounted for additional, unknown disturbances such as grazing by
using a block sampling design, and stratifying our statistical analyses
by these blocks. Long-term grazing data were not available. Therefore,
we assumed that plots within blocks were close enough together that they
had experienced similar grazing pressure. Additionally, we visually
assessed the impact of grazing on-site, aggregated what records we could
for the allotments in our study (billed animal unit months (AUM)
provided by the Bureau of Land Management), and normalized AUM by unit
area and included these data in our statistical modeling.

Once we constrained the area to a consistent sampling space, we used
Landsat-derived fire data to stratify the space along a fire frequency
gradient. To generate fire history maps, we first extracted only the
values two to four (low, medium and high severity) from each yearly burn
severity mosaic from the Monitoring Trends in Burn Severity project
(MTBS; Eidenshink et al. 2007), as these were the values where one can
be reasonably certain that they actually burned. Unburned patches and
post-fire green-up, which could be caused by a response to fire or an
unburned patch, were excluded. To generate fire frequency maps, we
reclassified each yearly layer to a binary grid, and summed all 31
layers. To avoid areas with less certain fire frequencies, we then
converted the MTBS fire perimeter polygons to layers of fire frequency
to extract only the grid cells where the frequency from the polygons
matched the frequency from the reclassified raster grid. To generate
last-year-burned maps, we reclassified each severity mosaic (values two
to four) to the fire year, and calculated the maximum year for the
entire time period for each pixel. To eliminate areas that had burned
more recently than 2014, we masked pixels that burned in 2015 according
to the MODIS MCD64 burned area product (Giglio et al. 2009).

Kolden et al. (2015) have brought up several shortcomings for the use of
the MTBS burn severity mosaics, in particular inconsistent development
of class thresholds and a lack of empirical relationships between the
classified values and ecological metrics. Because we only used these
data to get a more precise estimate of fire occurrence, (i.e.~we used it
to eliminate areas of uncertainty) rather than using the severity data
as an independent variable for analysis, we thought it sufficient to use
these data in this state. Another shortcoming that should be noted is
that there is no practical way for us to know what these sites looked
like before the earliest fires in the fire record. The fact that our
unburned control plots were all mature sagebrush is one piece of
evidence suggesting these sites were mature sagebrush pre-fire, but we
cannot be 100\% certain, and this is a shortcoming of all chronosequence
studies (Walker et al. 2010).

We selected seven blocks in our sampling space in accessible areas where
there was a range of fire frequencies and unburned areas for controls
within close proximity (0.5 - 10 km). Within each block, we created
spatially balanced random points (Theobald et al. 2007) for each fire
frequency, and sampled one plot for each fire history class within the
block. At each block, we first sampled the unburned control plot to
confirm that the area was indeed the correct vegetation type, and then
sampled burned plots. After navigating to the predetermined coordinates
for each plot we first confirmed the physical characteristics (soil
type, lack of obvious restoration, lack of obvious overgrazing) were
within the constraints of our sampling design. If a predetermined point
was not suitable (e.g.~soil was too rocky or sandy, an unburned control
plot had obviously burned, or it was the wrong ecological site type), we
referred to georeferenced PDFs of our fire history atlas that we
accessed with a simple application (Avenza Maps
\url{https://www.avenzamaps.com/}) on a mobile device and located nearby
areas within the site that were suitable. When a suitable area was found
we used a random number generator to pick a random bearing and a random
distance, and navigated to the new plot location.

We sampled 28 plots that fell along a gradient of fire frequency (zero
to three fires; N = seven plots per frequency) and a range of times
since fire (4-31 years; mean = 17.6, std = 6.6; Figure 2). Because most
of the fire effects research in this system has been done within five
years of a fire, we aimed to have the time since fire of all of the
plots greater than or equal to 5 years. We encountered 53 plant species
- 12 were introduced and 41 were native (Table S2).

\emph{Plot establishment}

We used GPS to navigate to predetermined plot locations. Upon arrival we
established a permanent marker at the southwest corner of the plot. We
recorded the slope, aspect, distance to the nearest \emph{A. tridentata}
individual or other shrub species, the topographic curvature of the site
(convex, concave, flat), evidence of ecological restoration, grazing
signs, and evidence of past fires. We then delineated a 50 x 50 meter
plot, and placed pin flags at 9 randomly determined
1m\textsuperscript{2} subplots within the plot with a minimum spacing of
3 meters. Pilliod and Arkle (2013) found this sampling density
sufficient for this ecosystem, if supplemental methods are used to
estimate disparate functional groups like trees and shrubs. Hence, we
used the point-quarter method as a supplement to estimate shrub cover
(see Pilliod \& Arkle (2013) for detailed methods).

\emph{Vegetation sampling}

To explore how fire frequency influences community composition and
diversity, we measured the occurrence and abundance of all species. We
identified and recorded occupancy data for every species within each
subplot, and took a photograph from nadir with an Olympus Stylus TG-870
digital camera to be analyzed later for percent cover.

We used `Samplepoint' software (Booth et al. 2006) to analyze the
digital photographs for percent cover. We prepared photographs for
analysis by cropping them to the 1 m x 1 m area of the subplot. Then we
used Samplepoint to overlay a regular grid of 100 points on each
picture, and at each point identified whether it was litter, bare
ground, rock, dung, or a plant. If it was a plant, we identified it to
species with the aid of the occupancy data recorded at the plot. These
data were then converted to percent cover. If we recorded a species as
present within the subplot, but it was missed by the photographic
analysis, we recorded it as 0.5\% cover.

\emph{Environmental data}

Aspect was converted to folded aspect (folded aspect = \textbar{}180 -
\textbar{}aspect - 225\textbar{}\textbar{}; McCune and Keon 2002). This
results in an approximation of heat load ranging from zero (northeast)
to 180 (southwest). Elevation was extracted from 10m resolution digital
elevation models. The study sites were situated among six grazing
allotments. To learn how climate before, during and after the fire event
affected the subsequent community composition and diversity, we
extracted monthly maximum vapor pressure deficit, maximum temperature,
and precipitation for the years before during and after the most recent
fire at each plot. Maximum temperature and maximum vapor pressure
deficit were averaged for the entire year before during and after, and
precipitation was averaged for the two winters (November - May) prior
and one after. We used monthly data provided by the PRISM climate group
(PRISM Climate Group 2016) for all climate variables. Variables used in
modeling are provided in Table 1. We also sampled soil C and N (see
Mahood (2017) for detailed methods).

\emph{Statistical analysis}

\emph{Community composition and environmental variables} To analyze how
fire frequency affects community composition, we used non-metric
multidimensional scaling (NMDS). We ran a rank correlation test for fire
history gradients against a matrix of relative cover of species per plot
to determine the best hierarchical clustering method for creating a
dissimilarity matrix. We used this index for NMDS to examine how those
fire history characteristics affected the floristic composition. To
assess which species and environmental variables had the most influence
on community composition, we added those variables to the ordinations
using the `envfit' function from Vegan, with 9,999 permutations and
stratified by the study block. Then we grouped species by their
biogeographical origin (i.e.~native or exotic), and used Tukey's test to
assess how fire frequency influenced native cover, exotic plant cover
and cheatgrass abundance.

\emph{Species richness, alpha diversity and beta diversity.} We created
species accumulation curves grouped by fire frequency to assess how fire
frequency affected species richness. This is different from alpha
diversity in that the species accumulation curve is estimating number of
species across all of the sites within each group with each added plot,
as opposed to simply calculating a diversity index for each plot. We
used the sample-based rarefaction method (Chiarucci et al. 2008, Oksanen
et al. 2016, R Core Team 2016). We used Tukey's Honest Significant
Differences test (hereafter, Tukey's test) to see if different fire
frequencies influenced alpha diversity (Shannon-weaver, Pielou evenness,
and number of species per plot). There are several ways to quantify beta
diversity, most of which are grouped into ``measures of continuity'' and
``measures of gain and loss'' (Koleff et al. 2003). We used the ``Z''
index and Whittaker's original beta diversity index for continuity
measures, and Simpson's index (based on G. Gaylord Simpson's asymmetric
index (Simpson 1943) and modified by Lennon et al. (2001), not to be
confused with Edward H. Simpson's Index (1949)) for a measure of gain
and loss. To see how beta diversity differed between fire frequencies,
we modeled the homogeneity of dispersion of those matrices (Anderson et
al. 2006), and ran pairwise permutation tests (Legendre et al. 2011) on
these models with 9,999 permutations, stratified by the study blocks. To
assess the influence of cheatgrass abundance on alpha and beta
diversity, we used linear mixed models (Pinheiro et al. 2017) with the
study block as a random effect. We included elevation as a fixed effect
in addition to cheatgrass due to its strong correlation with temperature
and moisture availability, and ecosystem resistance and resilience
(Chambers et al. 2014). We ensured that predictors had no
multicollinearity using a variable inflation factor test (Fox and
Weisberg 2011), and used the partial coefficient of determination
(Jaeger et al. 2016) to determine the cheatgrass component of the model.
To aid visualization, we removed the partial effects of elevation from
the dependent variables (Hohenstein and Kliegl 2018).

\emph{Modeling which fire and climate variables drive post-fire
composition and diversity.} To assess how pre- and post-fire climate,
along with soil and other environmental variables (Table 1) affected
post-fire community composition and diversity, we used permutational
multivariate analysis of variance (PERMANOVA). PERMANOVA uses a
dissimilarity matrix as the response variable, and columns from a
separate data frame as the predictors. It makes the assumption that
groups being modeled have homogeneous dispersions. If the test is run on
groups with heterogeneous dispersions, it is vulnerable to type 1 error
(Anderson and Walsh 2013). To account for this we built multivariate
homogeneity of groups dispersions (MHGD) models on our community
clustering and beta diversity matrices grouped by block, fire frequency,
and burned vs unburned. We then ran ANOVAs and Tukey's test on each
model, with p values below 0.05 considered to be an indication of
heterogeneous dispersions. After removing variables with
multi-collinearity, we built PERMANOVA models with both community
clustering and beta diversity matrices using an additive model-building
process, with 9,999 permutations and stratifying the permutations by the
study blocks, with the aim of producing parsimonious models.

\emph{Code availability}

Data and code to reproduce the analysis are available at
\url{https://www.github.com/admahood/ff_study}.

\textbf{Results}

\emph{Community composition fundamentally changed after one fire}

The rank index test showed the Kulczynski index to have the most
consistent high scores across gradients of fire history characteristics,
so we used this index for our hierarchical clustering and NDMS analyses.
NMDS (Non-metric fit, R\(^2\) = 0.992, Linear fit, R\(^2\) = 0.972)
showed 7 unburned plots clustered around high abundances of \emph{A.
tridentata}, and 18 burned plots clustered around \emph{B. tectorum}
(Figure 3). Two thrice-burned plots were dominated by exotic annual
forbs (\emph{Sisymbrium altissimum} and \emph{Erodium cicutarium}) and
one was dominated by the native perennial grass \emph{P. secunda} (these
are the three thrice-burned plots outside of the ``burned'' ellipse).
The ordination showed a clear separation between burned and unburned
plots, but fire frequency was not significantly correlated with the
ordination, nor were any environmental variables.

For the Tukey's tests of exotic versus native cover, there were
differences between unburned and burned plots (p \textless{} 0.05) for
both exotic (increased by 10\%) and native cover (decreased by 20\%),
and no differences among the burned plots. (Figure 4a-b). After dividing
the mean cover estimates into native and exotic life form groups (annual
and perennial graminoids and forbs, and shrubs), we saw lower native
shrub cover for burned plots fire (24\% to 3\%), coupled with higher
annual grass cover (4\% to 14\%; Figure 5).

\emph{Plant biodiversity decreased with each successive fire}

We found a decline in plant diversity at sites that had burned more
frequently. Species richness estimates declined as fire frequency
increased (Figure 6, Table S3). The number of species and the
Shannon-weaver index were higher in unburned plots, but the differences
were not significant, and Pielou evenness was not different between
frequencies (Figure 4c-e). All three indexes of Beta diversity followed
very similar patterns, so we only report on Whittaker's index here. It
was not different between zero and two fires, and lower for
thrice-burned plots (Figure 4f), meaning that there is less
dissimilarity within the group of thrice-burned plots, and more
dissimilarity within the other groupings.

\emph{Alpha diversity and evenness decreased with cheatgrass abundance}

Cheatgrass abundance had a negative relationship with Shannon-weaver
diversity (p \textless{}\textless{} 0.05, partial R\(^2\) = 0.65) and
Pielou evenness (p \textless{}\textless{} 0.05, partial R\(^2\) = 0.51),
a weak negative relationship with the number of species (p \textless{}
0.05, partial R\(^2\) = 0.24), and no relationship to beta diversity (p
\textgreater{} 0.5, partial R\(^2\) = 0.08; Figure 7, Table 2).
Elevation was important in all models except Pielou evenness (Table 2).

\emph{Different climate and fire variables predict post-fire compsition
and diversity}

PERMANOVA models showed that fire history and environmental factors
influenced community composition and beta diversity differently. ANOVAs
and Tukey's tests on MHGD models showed no heterogeneity in groups
dispersions for both beta diversity and hierarchical clustering (p
\textgreater{} 0.05 for all models). Community composition after fire
was most affected by fire frequency, time since fire, maximum vapor
pressure deficit of the year of the fire, and the interaction between
fire frequency and time since fire (Table 3, R\textsuperscript{2} =
0.55). The relatively low amounts of variation accounted for by the
individual variables indicates these are subtle effects. Beta diversity
on the other hand was influenced most by winter precipitation one and
two years prior to the fire, fire frequency, and the interaction between
winter precipitation one year prior and max temperature for the year
after the fire (Table 4, R\textsuperscript{2} = 0.62). Here, the effect
was more pronounced, as more variation accounted for by the three most
statistically significant variables (fire frequency and precipitation
one and two winters prior to the fire).

\textbf{Discussion}

The purpose of this study is to assess how Wyoming big sagebrush plant
communities respond to being burned repeatedly before returning to their
prior condition. The combination of a 32-year fire history atlas and the
use of the RRQRR (Theobald et al. 2007) to randomly stratify the
sampling blocks over a large area provides broad-scale statistical
inference for the lower elevation (\textless{}1500 m) portion of the
Wyoming big sagebrush ecosystem. These lower elevation sites generally
experience higher temperatures and lower soil moisture, and it is well
documented that they have lower resilience after wildfires (Chambers et
al. 2014). We did not detect recovery of Wyoming big sagebrush at our
sites, and also found that while the cover of Bromus tectorum does not
change with successive fires, the number of species in the species pool
does decrease, and that biodiversity decreases with cover of \emph{B.
tectorum}. The results of this study may seem to conflict with other
recent studies documenting Wyoming big sagebrush sagebrush recovery in
the Great Basin (Ellsworth et al. 2016, e.g. Shinneman and McIlroy
2016). But all of the studies we are aware of showing sagebrush recovery
were conducted at cooler, wetter sites, where Wyoming big sagebrush is
more resilient after fire (Chambers et al. 2014).

Coupling the 30+ year fire history atlas created here with intensive
field sampling offers a unique opportunity to explore plant diversity
and composition changes in areas that have relatively high fire
frequencies, such as grass-dominated or grass-invaded areas (Balch et
al. 2013). As annual grass invasions and their alterations to fire
regimes are a global phenomenon (D'Antonio and Vitousek 1992, Brooks et
al. 2004), this type of study design will be useful for understanding
the consequences of changing fire regimes in other regions.
Additionally, new algorithms are being developed that will lead to more
accurate and precise fire data products (e.g. Hawbaker et al. 2015),
leading to more nuanced fire history atlases, and thus more precise
sampling stratifications -- especially now that burn severity
information can be easily incorporated (Eidenshink et al. 2007).

\emph{Community composition fundamentally changes after one fire}

In lower elevation \emph{A. tridendata} ssp. \emph{wyomingensis}
systems, our results show that one fire can convert this shrub-dominated
system to one composed mainly of introduced annual grasses and forbs,
and we demonstrate that this new state can persist for decades with
little sign of recovery to its prior condition. While almost all of our
burned plots were dominated by cheatgrass, several thrice-burned plots
were dominated by \emph{P. secunda} or exotic annual forbs (see figure
3, where there are 3 plots that are outside the confidence envelope
containing all other burned plots). This corroborates previous work
showing that fire can push cheatgrass-invaded grassland and shrubland
communities into those dominated by cheatgrass, \emph{P. secunda}, and
exotic forbs, while uninvaded sites, or sites that are invaded but still
have significant bunchgrass communities, can persist in a state of
native bunchgrasses and forbs (Davies et al. 2012, Reisner et al. 2013,
Condon and Pyke 2018). Other studies have found that topography can be a
mediating factor, with native bunchgrasses more likely to persist on
steeper, more north-facing slopes in the face of invasion and
disturbance (Rodhouse et al. 2014, Reed-Dustin et al. 2016). One
hypothesis that we were not able to test in this study is that
increasing fire frequency may select for more fire-resilient plant
functional traits. More research is needed to investigate the
relationship between fire frequency and functional traits. While it has
been demonstrated that \emph{B. tectorum} establishes immediately
post-fire and can persist in the shorter term (Davies et al. 2012, Hanna
and Fulgham 2015), we show that this novel grass state can persist for
long periods (i.e. \textgreater{} 17 years), corroborating recent work
(Reed-Dustin et al. 2016). If there was recovery our study design would
have enabled us to detect it, as Wyoming big sagebrush has been found to
recover from disturbance in as little as nine (Wambolt et al. 2001) to
20 years (Shinneman and McIlroy 2016) following fire, and our fire
history atlas goes back 32 years.

\emph{Biodiversity decreases with each subsequent fire}

Here we show that over a three decade period repeated fires had
long-lasting effects on community composition and biodiversity in
Wyoming big sagebrush ecosystems. Species richness declined with
increasing fire frequency, but measures of alpha and beta diversity
decreased after one and three fires, respectively (Figure 4a-b). Species
accumulation curves demonstrated that repeated fires are decreasing the
overall pool of species from which an individual patch might draw from.
So while there may not have been significant differences in alpha
diversity as fire frequency increased, as the number of species each
plot can draw from decreased, this signal manifested itself when beta
diversity declined after three fires.

We found negative relationships between cheatgrass abundance and alpha
and beta diversity, as we hypothesized, but no relationship between
cheatgrass abundance and the number of fires. Establishment and
dominance of cheatgrass after fire is well documented (Whisenant 1990,
Balch et al. 2013), and the relationship between fire and species
richness is clear from this work. This implies that once an area is
invaded by cheatgrass, the competitive effects from its increased
abundance combine with its effect on fire frequency to exclude species
that either cannot compete for moisture or cannot survive fire. It
should be noted that because we selected sites that had burned at least
three times since 1984, we may have biased our results to be applicable
to only those areas that are susceptible to initiating a grass-fire
cycle.

\emph{Time since fire and vapor pressure deficit drive community
composition}

PERMANOVA models showed that fire history and climate variables affect
diversity and community composition differently. Composition was found
to be influenced by both fire frequency and time since fire, and high
vapor pressure deficit the year of the fire. This suggests that drought
stress exerts a significant influence on the particular plant species
that will survive and persist after a fire, and this effect can still be
detected decades after the fire burned. Shinneman and McIlroy (2016)
also found that climatic variables around the time of the fire influence
the eventual composition, namely winter precipitation the year after the
fire was beneficial for sagebrush recovery, but winter precipitation 2
years later had a negative effect. Elevation and recovery have been
shown to be positively related in this system (Knutson et al. 2014), and
most of the studies showing fast recovery times were done at higher
elevations and latitudes (Wambolt et al. 2001, Hanna and Fulgham 2015,
Ellsworth et al. 2016), in areas with long-term grazing exclusion
(Ellsworth et al. 2016), or on sites that were specifically selected
because their topographic position was such that there was potential for
sagebrush recovery (Shinneman and McIlroy 2016). Here, we found that on
low elevation sites, even after an average of 17 years, post-fire
sagebrush cover was very low (\textless{}6\%; also see Reed-Dustin et
al. 2016). These differences in recovery rates (i.e.~9-20 years at
cooler sites vs no detectable recovery at hotter sites) could be due to
a slowing down of recovery rates as the system loses resilience with
increasing drought stress at hotter sites, while cooler sites have not
yet experienced sufficient drought stress to hamper recovery
(\emph{sensu} Leemput et al. 2017).

\emph{Fire frequency and antecedent precipitation drive beta diversity}

Beta diversity was most heavily influenced by fire frequency,
precipitation for the two wet seasons prior to the fire, and an
interaction between antecedent precipitation and maximum temperature for
the year after the fire. Antecedent precipitation has been shown in
other studies to be an important predictor of fire occurrence and burned
area in this system (Abatzoglou and Kolden 2013, Balch et al. 2013).
Since this is a fuel-limited system, high precipitation increases fine
fuel loads and continuity (Davies and Nafus 2013), leading to higher
fire probability, more homogeneously burning fires, and larger extents.
Increased fine fuel loads could also be the driving factor behind
decreasing diversity. Following highly contiguous and extensive fires
there would be fewer unburned patches as seed sources which are
essential for the seed-obligate sagebrush to reestablish quickly
(Shinneman and McIlroy 2016). In addition, Wyoming big sagebrush is an
opportunist in reproduction, setting most of its seed in wet years
(Meyer 1994) during the short window in early spring when enough water
is available in the soil for plants to uptake nutrients (Ryel et al.
2010, Schlaepfer et al. 2014). So, in the years that Wyoming big
sagebrush is maximizing its expenditure on reproductive resources,
increased horizontal fuel continuity of invasive annual grasses (Davies
and Nafus 2013) a) increases the probability of burning, and b)
increases interspecific competition for resources post-fire. This may
result in a more homogeneous post-fire landscape populated mostly by
fire-tolerant plants.

\emph{Management implications}

This work adds to the existing body of literature that suggests that in
low elevation (\textless{} 1700m) Wyoming big sagebrush systems wildfire
should be minimized due to the negative effects of single and repeated
fires on community composition and biodiversity. The reality is that
wildfire can not be prevented, but fire suppression policies and
practices could be crafted to maximize the number and size of unburned
patches within burns to increase the probability that Wyoming big
sagebrush and other native seed-obligates recover post-fire. These
results also imply that prescribed burning is a risky proposition with
potentially disastrous consequences for biodiversity and ecosystem
structure and function. However, we did not directly assess the
influence of prescribed fires in this study. Prescribed fires typically
are conducted at a cooler time of year outside of or at the shoulder of
the fire season, and may have different ecological effects due to the
phenological stage plants would be in at this different time of year, as
well as the lower burn severity that would be expected due to cooler
ambient air temperatures and higher soil moisture. At a cooler, wetter
site where grazing has been excluded since 1994, Ellsworth et al. (2016)
detected the recovery of sagebrush 17 years after prescribed fires were
conducted in late September 1997, which is the natural end of the fire
season at that location. Two other studies at higher latitudes concluded
that prescribed burning to be an unwise action even at those wetter
sites. Beck et al. (2009) studied an area in southeast Idaho that was
burned in late August 1989 by prescribed fire 14 years post-fire for its
utility in improving sage grouse habitat. They recommended against
prescribed fires due to the lack of recovery of sagebrush. Wambolt et
al. (2001) found minimal benefit to the herbaceous plant community at 13
sites that had burned in prescribed fires in Western Montana, with
little shrub recovery 6-15 years after fire. Thus, there is conflicting
evidence on the use of prescribed fires for management objectives even
at cooler wetter sites, providing less optimism for the use of
prescribed fires in the lower elevation portion of the Wyoming big
sagebrush ecosystems studied here. Future research could focus on
comparing low elevation Wyoming big sagebrush sites that have been
burned in prescribed fires in the past paired with nearby areas that
burned in wildfires, with particular emphasis on teasing out the effects
of seasonality and burn severity.

Our results from PERMANOVA modeling suggest that the success of
post-fire restoration efforts will depend not only on elevation and
topographic conditions (Arkle et al. 2014), but also the climatic
conditions that occur around the time of the fire. This could mean that
in a very dry year less money is spent on restoration efforts on low
elevation sites, focusing instead on higher elevation sites and cooler
aspects, and in wet years directing more funding towards those more
vulnerable low elevation, southwest-facing sites.

Disagreement on the actual historical fire rotation limits our ability
to determine if Wyoming big sagebrush is fire-sensitive or
fire-resistant. However, this question may be irrelevant given the
disruption and interaction between invasive annual grasses and fires. We
demonstrate that when both fire and invasive annual grasses operate in
conjunction, sagebrush is fire-sensitive. Moreover, we show that an
alternate exotic grass state can persist for 17 years post-fire even
with only a single burn. This makes the use of prescribed burning
problematic, as the risk of a fire-prone grassland establishing after a
fire likely outweighs the potential benefits of a prescribed fire. Our
results are specific to lower elevation (\textless{} 1700m), dryer,
hotter Wyoming big sagebrush sites, and it remains to be explored how
sagebrush at higher elevations and latitudes responds to increasing fire
frequency, and how it will respond under future climate change
scenarios. However, if temperatures continue to rise as projected in
this region (Garfin et al. 2014), those areas may also become
susceptible to a strong grass-fire cycle. Overall, this effort
demonstrates that sagebrush communities are vulnerable to repeated fires
(Seipel et al. 2018), which should be taken into account in land
management decisions (Chambers et al. 2017) that attempt to conserve or
restore these valuable ecosystems, and the threatened species that they
harbor.

\textbf{Acknowledgements}

We are grateful for the assistance of Nick Whittemore and Kathleen
Weimer for their assistance in the field and in the lab. We thank Tom
Veblen and Carson Farmer for comments on previous versions of the
manuscript. We appreciated the constructive criticism from three
anonymous reviewers which greatly improved the paper. Thanks to Max
Joseph for help with the data analysis. We are also very grateful for
the support of the Nevada Bureau of Land Management, and the Central
Nevada Interagency Dispatch Center. This work was funded by the National
Aeronautics and Space Administration Terrestrial Ecology Program under
Award NNX14AJ14G, and start-up funding from the Department of Geography
at CU Boulder.

\textbf{Literature Cited} \singlespacing

\hypertarget{refs}{}
\hypertarget{ref-Abatzoglou2013}{}
Abatzoglou, J. T., and C. A. Kolden. 2013. Relationships between climate
and macroscale area burned in the western United States. International
Journal of Wildland Fire 22:1003--1020.

\hypertarget{ref-Amundson1997}{}
Amundson, R., and H. Jenny. 1997. On a state factor model of ecosystems.
BioScience 47:536--543.

\hypertarget{ref-Anderson2001}{}
Anderson, J. E., and R. S. Inouye. 2001. Landscape-Scale Changes in
Plant Species Abundance and Biodiversity of a Sagebrush Steppe over 45
Years. Ecological Monographs 71:531--556.

\hypertarget{ref-Anderson2013}{}
Anderson, M. J., and D. C. I. Walsh. 2013. PERMANOVA , ANOSIM , and the
Mantel test in the face of heterogeneous dispersions: What null
hypothesis are you testing? Ecological Monographs 83:557--574.

\hypertarget{ref-Anderson2006}{}
Anderson, M. J., K. E. Ellingsen, and B. H. McArdle. 2006. Multivariate
dispersion as a measure of beta diversity. Ecology Letters 9:683--693.

\hypertarget{ref-Arkle2014}{}
Arkle, R., D. Pilliod, S. Hanser, M. L. Brooks, J. C. Chambers, J. B.
Grace, K. C. Knutson, D. A. Pyke, J. L. Welty, and T. A. Wirth. 2014.
Quantifying restoration effectiveness using multi-scale habitat models:
implications for sage-grouse in the Great Basin. Ecosphere 5:1--32.

\hypertarget{ref-Bagchi2013}{}
Bagchi, S., D. D. Briske, B. T. Bestelmeyer, and X. Ben Wu. 2013.
Assessing resilience and state-transition models with historical records
of cheatgrass Bromus tectorum invasion in North American
sagebrush-steppe. Journal of Applied Ecology 50:1131--1141.

\hypertarget{ref-Baker2006}{}
Baker, W. L. 2006. Fire and restoration of sagebrush ecosystems.
Wildlife Society Bulletin 34:177--185.

\hypertarget{ref-Balch2017}{}
Balch, J. K., B. A. Bradley, J. T. Abatzoglou, R. C. Nagy, E. J. Fusco,
and A. L. Mahood. 2017. Human-started wildfires expand the fire niche
across the United States. Proceedings of the National Academy of
Science.

\hypertarget{ref-Balch2013}{}
Balch, J. K., B. A. Bradley, C. M. D'Antonio, and J. Gómez-Dans. 2013.
Introduced annual grass increases regional fire activity across the arid
western USA (1980-2009). Global Change Biology 19:173--183.

\hypertarget{ref-Bansal2016}{}
Bansal, S., and R. L. Sheley. 2016. Annual grass invasion in sagebrush
steppe: the relative importance of climate, soil properties and biotic
interactions. Oecologia 181:543--557.

\hypertarget{ref-Beck2009}{}
Beck, J. L., J. W. Connelly, and K. P. Reese. 2009. Recovery of Greater
Sage-Grouse habitat features in Wyoming big sagebrush following
prescribed fire. Restoration Ecology 17:393--403.

\hypertarget{ref-Booth2006}{}
Booth, D. T., S. E. Cox, and R. D. Berryman. 2006. Point sampling
digital imagery with 'Samplepoint'. Environmental Monitoring and
Assessment 123:97--108.

\hypertarget{ref-Bowman2011}{}
Bowman, D. M. J. S., J. K. Balch, P. Artaxo, W. J. Bond, M. A. Cochrane,
C. M. D'Antonio, R. S. Defries, F. H. Johnston, J. E. Keeley, M. A.
Krawchuk, C. A. Kull, M. Mack, M. A. Moritz, S. J. Pyne, C. I. Roos, A.
C. Scott, N. S. Sodhi, and T. W. Swetnam. 2011. The human dimension of
fire regimes on Earth. Journal of Biogeography 38:2223--2236.

\hypertarget{ref-Bradley2008}{}
Bradley, B. A., and J. F. Mustard. 2008. Comparison of phenology trends
by land cover class: A case study in the Great Basin, USA. Global Change
Biology 14:334--346.

\hypertarget{ref-Brooks2004b}{}
Brooks, M. L., C. M. D. Antonio, D. M. Richardson, J. B. Grace, J. E.
Keeley, J. M. DiTomaso, R. J. Hobbs, M. Pellant, and D. Pyke. 2004.
Effects of Invasive Alien Plants on Fire Regimes. BioScience
54:677--688.

\hypertarget{ref-Brooks2015}{}
Brooks, M. L., J. R. Matchett, D. J. Shinneman, and P. S. Coates. 2015.
Fire Patterns in the Range of the Greater Sage-Grouse , 1984 -- 2013 ---
Implications for Conservation and Management: U.S. Geological Survey
Open-File Report 2015-1167. Page 66.

\hypertarget{ref-Bukowski2013}{}
Bukowski, B., and W. L. Baker. 2013. Historical fire regimes,
reconstructed from land-survey data, led to complexity and fluctuation
in sagebrush landscapes. Ecological Applications 23:546--564.

\hypertarget{ref-Chambers2017}{}
Chambers, J. C., J. L. Beck, J. B. Bradford, J. Bybee, S. Campbell, J.
Carlson, T. J. Christiansen, K. J. Clause, G. Collins, M. R. Crist, J.
B. Dinkins, K. E. Doherty, F. Edwards, S. Espinosa, K. A. Griffin, P.
Griffin, J. R. Haas, S. E. Hanser, D. W. Havlina, K. F. Henke, J. D.
Hennig, L. A. Joyce, F. F. Kilkenny, S. M. Kulpa, L. L. Kurth, J. D.
Maestas, M. Manning, K. E. Mayer, B. A. Mealor, C. McCarthy, M. Pellant,
M. A. Perea, K. L. Prentice, D. A. Pyke, L. A. Wiechman, and A.
Wuenschel. 2017. Science Framework for Conservation and Restoration of
the Sagebrush Biome: Linking the Department of the Interior's Integrated
Rangeland Fire Management Strategy to Long-Term Strategic Conservation
Actions . Part 1. Science basis and applications. Gen. Te:213.

\hypertarget{ref-Chambers2014}{}
Chambers, J. C., B. A. Bradley, C. S. Brown, C. D'Antonio, M. J.
Germino, J. B. Grace, S. P. Hardegree, R. F. Miller, and D. A. Pyke.
2014. Resilience to stress and disturbance, and resistance to Bromus
tectorum L. invasion in cold desert shrublands of western North America.
Ecosystems 17:360--375.

\hypertarget{ref-Chambers2007}{}
Chambers, J. C., B. A. Roundy, R. R. Blank, S. E. Meyer, and A.
Whittaker. 2007. What makes Great Basin sagebrush ecosystems invasible
by Bromus tectorum? Ecological Monographs 77:117--145.

\hypertarget{ref-Chiarucci2008}{}
Chiarucci, A., G. Bacaro, D. Rocchini, and L. Fattorini. 2008.
Discovering and rediscovering the sample-based rarefaction formula in
the ecological literature. Community Ecology 9:121--123.

\hypertarget{ref-Coates2016}{}
Coates, P. S., M. A. Ricca, B. G. Prochazka, M. L. Brooks, K. E.
Doherty, T. Kroger, E. J. Blomberg, C. A. Hagen, and M. L. Casazza.
2016. Wildfire, climate, and invasive grass interactions negatively
impact an indicator species by reshaping sagebrush ecosystems.
Proceedings of the National Academy of Sciences:201606898.

\hypertarget{ref-Condon2018}{}
Condon, L. A., and D. A. Pyke. 2018. Fire and Grazing Influence Site
Resistance to Bromus tectorum Through Their Effects on Shrub, Bunchgrass
and Biocrust Communities in the Great Basin (USA). Ecosystems
21:1416--1431.

\hypertarget{ref-Davies2012}{}
Davies, G. M., J. D. Bakker, E. Dettweiler-Robinson, P. W. Dunwiddie, S.
A. Hall, J. Downs, and J. Evans. 2012. Trajectories of change in
sagebrush steppe vegetation communities in relation to multiple
wildfires. Ecological Applications 22:1562--1577.

\hypertarget{ref-Davies2011}{}
Davies, K. W. 2011. Plant community diversity and native plant abundance
decline with increasing abundance of an exotic annual grass. Oecologia
167:481--491.

\hypertarget{ref-Davies2013}{}
Davies, K. W., and A. M. Nafus. 2013. Exotic annual grass invasion
alters fuel amounts, continuity and moisture content. International
Journal of Wildland Fire 22:353--358.

\hypertarget{ref-Davies2011a}{}
Davies, K. W., C. S. Boyd, J. L. Beck, J. D. Bates, T. J. Svejcar, and
M. A. Gregg. 2011. Saving the sagebrush sea: An ecosystem conservation
plan for big sagebrush plant communities. Biological Conservation
144:2573--2584.

\hypertarget{ref-Davies2009}{}
Davies, K. W., T. J. Svejcar, and J. D. Bates. 2009. Interaction of
Historical and Nonhistorical Disturbances Maintains Native Plant
Communities. Ecological Applications 19:1536--1545.

\hypertarget{ref-Dennison2014}{}
Dennison, P. E., S. C. Brewer, J. D. Arnold, and M. A. Moritz. 2014.
Large wildfire trends in the western United States, 1984--2011.
Geophysical Research Letters:2928--2933.

\hypertarget{ref-DAntonio1992}{}
D'Antonio, C. M., and P. M. Vitousek. 1992. Biological invasions by
exotic grasses, the grass/fire cycle, and global change. Annual Review
of Ecological Systems 23:63--87.

\hypertarget{ref-Eidenshink2007}{}
Eidenshink, J., B. Schwind, K. Brewer, Z.-l. Zhu, B. Quayle, and S.
Howard. 2007. A Project for Monitoring Trends in Burn Severity. Fire
Ecology 3:3--21.

\hypertarget{ref-Ellsworth2013}{}
Ellsworth, L. M., and J. B. Kauffman. 2013. Seedbank responses to spring
and fall prescribed fire in mountain big sagebrush ecosystems of
differing ecological condition at Lava Beds National Monument,
California. Journal of Arid Environments 96:1--8.

\hypertarget{ref-Ellsworth2016}{}
Ellsworth, L. M., D. W. Wrobleski, J. B. Kauffman, and S. A. Reis. 2016.
Ecosystem resilience is evident 17 years after fire in Wyoming big
sagebrush ecosystems. Ecosphere 7.

\hypertarget{ref-Fox2011}{}
Fox, J., and S. Weisberg. 2011. An R companion to applied regression.
Second. Sage, Thousand Oaks CA.

\hypertarget{ref-Garfin2014}{}
Garfin, G., G. Franco, H. Blanco, A. Comrie, P. Gonzalez, T. Piechota,
R. Smyth, and R. Waskom. 2014. Southwest: The Third National Climate
Assessment. In J. M. Melillo, T. C. Richmond, \& G. W. Yohe (Eds.),
Climate Change Impacts in the United States: The Third National Climate
Assessment (pp. 462-486). Pages 462--486.

\hypertarget{ref-Gasch2013}{}
Gasch, C. K., S. F. Enloe, P. D. Stahl, and S. E. Williams. 2013. An
aboveground -- belowground assessment of ecosystem properties associated
with exotic annual brome invasion. Biology and Fertility of Soils
49:919--928.

\hypertarget{ref-Giglio2009}{}
Giglio, L., T. Loboda, D. P. Roy, B. Quayle, and C. O. Justice. 2009. An
active-fire based burned area mapping algorithm for the MODIS sensor.
Remote Sensing of Environment 113:408--420.

\hypertarget{ref-Hanna2015}{}
Hanna, S. K., and K. O. Fulgham. 2015. Post-fire vegetation dynamics of
a sagebrush steppe community change significantly over time. California
Agriculture 69:36--42.

\hypertarget{ref-Hawbaker2015a}{}
Hawbaker, T. J., S. Stitt, Y.-J. Beal, G. Schmidt, J. Falgout, B.
Williams, and J. Takacs. 2015. Provisional burned area essential climate
variable (BAECV) algorithm description. United States Geological Survey.

\hypertarget{ref-Hohenstein2018}{}
Hohenstein, S., and R. Kliegl. 2018. Remef: Remove partial effects.

\hypertarget{ref-IRMT2016}{}
Integrated Rangeland Fire Management Strategy Actionable Science Plan
Team. 2016. The Integrated Rangeland Fire Management Strategy Actionable
Science Plan. Page 128. U.S. Department of the Interior, Washington, DC.

\hypertarget{ref-Jaeger2017}{}
Jaeger, B. C., L. J. Edwards, K. Das, and P. K. Sen. 2016. An
\textless{}i\textgreater{}R\textless{}/i\textgreater{}
\textless{}sup\textgreater{}2\textless{}/sup\textgreater{} statistic for
fixed effects in the generalized linear mixed model. Journal of Applied
Statistics 44:1086--1105.

\hypertarget{ref-Jolly2015}{}
Jolly, W. M., M. a. Cochrane, P. H. Freeborn, Z. A. Holden, T. J. Brown,
G. J. Williamson, and D. M. J. S. Bowman. 2015. Climate-induced
variations in global wildfire danger from 1979 to 2013. Nature
Communications 6:7537.

\hypertarget{ref-Knutson2014}{}
Knutson, K. C., D. A. Pyke, T. A. Wirth, R. S. Arkle, D. S. Pilliod, M.
L. Brooks, J. C. Chambers, and J. B. Grace. 2014. Long-term effects of
seeding after wildfire on vegetation in Great Basin shrubland
ecosystems. Journal of Applied Ecology 51:1414--1424.

\hypertarget{ref-Kolden2015}{}
Kolden, C. A., A. M. S. Smith, and J. T. Abatzoglou. 2015. Limitations
and utilisation of Monitoring Trends in Burn Severity products for
assessing wildfire severity in the USA. International Journal of
Wildland Fire 24:1023--1028.

\hypertarget{ref-Koleff2003}{}
Koleff, P., K. J. Gaston, and J. J. Lennon. 2003. Measuring beta
diversity for presence-absence data. Journal of Animal Ecology
72:367--382.

\hypertarget{ref-Krawchuk2009a}{}
Krawchuk, M. A., M. A. Moritz, M. A. Parisien, J. Van Dorn, and K.
Hayhoe. 2009. Global pyrogeography: The current and future distribution
of wildfire. PLoS ONE 4.

\hypertarget{ref-VandeLeemput2017}{}
Leemput, I. A. van de, V. Dakos, M. Scheffer, and E. H. van Nes. 2017.
Slow Recovery from Local Disturbances as an Indicator for Loss of
Ecosystem Resilience. Ecosystems:1--12.

\hypertarget{ref-Legendre2011}{}
Legendre, P., J. Oksanen, and C. J. F. ter Braak. 2011. Testing the
significance of canonical axes in redundancy analysis. Methods in
Ecology and Evolution 2:269--277.

\hypertarget{ref-Lennon2001}{}
Lennon, J. J., P. Koleff, J. Greenwood, and K. J. Gaston. 2001. The
geographical structure of british bird distributions: Diversity, spatial
turnover and scale. Journal of Animal Ecology 70:966--979.

\hypertarget{ref-Lesica2007}{}
Lesica, P., S. V. Cooper, and G. Kudray. 2007. Recovery of Big Sagebrush
Following Fire in Southwest Montana. Rangeland Ecology \& Management
60:261--269.

\hypertarget{ref-Liu2013}{}
Liu, Y., S. L. Goodrick, and J. A. Stanturf. 2013. Future U.S. wildfire
potential trends projected using a dynamically downscaled climate change
scenario. Forest Ecology and Management 294:120--135.

\hypertarget{ref-Mahood2017}{}
Mahood, A. L. 2017. Long-Term Effects of Repeated Fires on the Diversity
and Composition of Great Basin Sagebrush Plant Communities. PhD thesis,
University of Colorado Boulder.

\hypertarget{ref-McCune2002}{}
McCune, B., and D. Keon. 2002. Equations for potential annual direct
incident radiation and heat load. Journal of Vegetation Science
13:603--606.

\hypertarget{ref-Meyer1994}{}
Meyer, S. E. 1994. Germination and establishment ecology of big
sagebrush: implications for community restoration. Pages 244--251
\emph{in} Symposium on management, ecology, and restoration of
lntermountain annual rangelands, boise, id, may 18-21, 1992.

\hypertarget{ref-Miller2013}{}
Miller, R. F., J. C. Chambers, D. A. Pyke, F. B. Pierson, C. J.
Williams, B. Fred, and C. A. Jason. 2013. A Review of Fire Effects on
Vegetation and Soils in the Great Basin Region: Response and Ecological
Site Characteristics. United States Department of Agriculture Forest
Service:1--136.

\hypertarget{ref-Mitchell2016}{}
Mitchell, R. M., J. D. Bakker, J. B. Vincent, and G. M. Davies. 2016.
Relative importance of abiotic, biotic, and disturbance drivers of plant
community structure in the sagebrush steppe. Ecological
Applications:n/a--n/a.

\hypertarget{ref-Moritz2012a}{}
Moritz, M. A., M.-A. Parisien, E. Batllori, M. A. Krawchuk, J. Van Dorn,
D. J. Ganz, and K. Hayhoe. 2012. Climate change and disruptions to
global fire activity. Ecosphere 3:49.

\hypertarget{ref-Vegan2016}{}
Oksanen, J., F. G. Blanchet, M. Friendly, R. Kindt, P. Legendre, D.
McGlinn, P. R. Minchin, R. B. O'Hara, G. L. Simpson, P. Solymos, M. H.
H. Stevens, E. Szoecs, and H. Wagner. 2016. Vegan: Community ecology
package.

\hypertarget{ref-Pilliod2013}{}
Pilliod, D. S., and R. S. Arkle. 2013. Performance of Quantitative
Vegetation Sampling Methods Across Gradients of Cover in Great Basin
Plant Communities. Rangeland Ecology \& Management 66:634--647.

\hypertarget{ref-ltdl2013}{}
Pilliod, D. S., and J. L. Welty. 2013. Land Treatment Digital Library,
https://ltdl.wr.usgs.gov/. Reston, VA.

\hypertarget{ref-Pilliod2017}{}
Pilliod, D. S., J. L. Welty, and R. S. Arkle. 2017. Refining the
cheatgrass-fire cycle in the Great Basin: Precipitation timing and fine
fuel composition predict wildfire trends. Ecology and Evolution:1--26.

\hypertarget{ref-Pinheiro2017}{}
Pinheiro, J., D. Bates, S. DebRoy, D. Sarkar, and R Core Team. 2017.
nlme: Linear and nonlinear mixed effects models.

\hypertarget{ref-PRISM2016}{}
PRISM Climate Group. 2016. PRISM Gridded Climate Data. Oregon State
University.

\hypertarget{ref-R}{}
R Core Team. 2016. R: A language and environment for statistical
computing. R Foundation for Statistical Computing, Vienna, Austria.

\hypertarget{ref-Reed-dustin2016}{}
Reed-Dustin, C. M., R. Mata-González, and T. J. Rodhouse. 2016.
Long-Term Fire Effects on Native and Invasive Grasses in Protected Area
Sagebrush Steppe. Rangeland Ecology \& Management 69:257--264.

\hypertarget{ref-Reisner2013}{}
Reisner, M. D., J. B. Grace, D. A. Pyke, and P. S. Doescher. 2013.
Conditions favouring Bromus tectorum dominance of endangered sagebrush
steppe ecosystems. Journal of Applied Ecology 50:1039--1049.

\hypertarget{ref-Rodhouse2014}{}
Rodhouse, T. J., K. M. Irvine, R. L. Sheley, B. S. Smith, S. Hoh, D. M.
Esposito, and R. Mata-Gonzalez. 2014. Predicting foundation bunchgrass
species abundances: sagebrush steppe. Ecosphere 108:1--16.

\hypertarget{ref-Rollins2009}{}
Rollins, M. G. 2009. LANDFIRE: a nationally consistent vegetation,
wildland fire, and fuel assessment. International Journal of Wildland
Fire 18:235--249.

\hypertarget{ref-Rowland2006}{}
Rowland, M. M., M. J. Wisdom, L. H. Suring, and C. W. Meinke. 2006.
Greater sage-grouse as an umbrella species for sagebrush-associated
vertebrates. Biological Conservation 129:323--335.

\hypertarget{ref-Ryel2010}{}
Ryel, R. J., A. J. Leffler, C. Ivans, M. S. Peek, and M. M. Caldwell.
2010. Functional differences in water-use patterns of contrasting life
forms in Great Basin steppelands. Vadose Zone Journal 9:548.

\hypertarget{ref-Schlaepfer2014}{}
Schlaepfer, D. R., W. K. Lauenroth, and J. B. Bradford. 2014. Natural
Regeneration Processes in Big Sagebrush (Artemisia tridentata).
Rangeland Ecology \& Management 67:344--357.

\hypertarget{ref-Schmidt2002}{}
Schmidt, K. M., J. P. Menakis, C. C. Hardy, W. J. Hann, and D. L.
Bunnell. 2002. Development of course-scale spatial data for wildland
fire and fuel management. Page 41 p. United States Department of
Agriculture Forest Service.

\hypertarget{ref-Seipel2018}{}
Seipel, T., L. J. Rew, K. T. Taylor, B. D. Maxwell, and E. A. Lehnhoff.
2018. Disturbance type influences plant community resilience and
resistance to Bromus tectorum invasion in the sagebrush steppe. Applied
Vegetation Science:1--10.

\hypertarget{ref-Shinneman2009}{}
Shinneman, D. J., and W. L. Baker. 2009. Environmental and climatic
variables as potential drivers of post-fire cover of cheatgrass (Bromus
tectorum) in seeded and unseeded semiarid ecosystems. International
Journal of Wildland Fire 18:191--202.

\hypertarget{ref-Shinneman2016}{}
Shinneman, D. J., and S. K. McIlroy. 2016. Identifying key climate and
environmental factors affecting rates of post-fire big sagebrush
(Artemisia tridentata) recovery in the northern Columbia Basin, USA.
International Journal of Wildland Fire 25:933--945.

\hypertarget{ref-Simpson1949}{}
Simpson, E. H. 1949. Measurement of diversity. Nature 163.

\hypertarget{ref-Simpson1943}{}
Simpson, G. G. 1943. Mammals and the nature of continents. American
Journal of Science 241:1--31.

\hypertarget{ref-wss2016}{}
Soil Survey Staff, Natural Resources Conservation Service, United States
Department of Agriculture (USDA). 2016. Web Soil Survey.

\hypertarget{ref-Theobald2007}{}
Theobald, D. M., D. L. Stevens, D. White, N. S. Urquhart, A. R. Olsen,
and J. B. Norman. 2007. Using GIS to Generate Spatially Balanced Random
Survey Designs for Natural Resource Applications. Environmental
Management 40:134--146.

\hypertarget{ref-Veen2018}{}
Veen, G. F., W. H. van der Putten, and T. M. Bezemer. 2018.
Biodiversity-ecosystem functioning relationships in a long-term
non-weeded field experiment. Ecology 99:1836--1846.

\hypertarget{ref-Walker2010}{}
Walker, L. R., D. A. Wardle, R. D. Bardgett, and B. D. Clarkson. 2010.
The use of chronosequences in studies of ecological succession and soil
development. Journal of Ecology 98:725--736.

\hypertarget{ref-Wambolt2001}{}
Wambolt, C. L., K. S. Walhof, and M. R. Frisina. 2001. Recovery of big
sagebrush communities after burning in south-western Montana. Journal of
environmental management 61:243--252.

\hypertarget{ref-West2002}{}
West, N. E., and T. P. Yorks. 2002. Vegetation responses following
wildfire on grazed and ungrazed sagebrush semi-desert. Journal of Range
Management 55:171--181.

\hypertarget{ref-Westerling2016}{}
Westerling, A. L. 2016. Increasing western US forest wildfire activity:
sensitivity to changes in the timing of spring. Philosophical
Transactions of the Royal Society B: Biological Sciences 371.

\hypertarget{ref-Westerling2006}{}
Westerling, A. L., H. G. Hidalgo, D. R. Cayan, and T. W. Swetnam. 2006.
Warming and earlier spring increase western U.S. forest wildfire
activity. Science 313:940--943.

\hypertarget{ref-Whisenant1990}{}
Whisenant, S. G. 1990. Changing fire frequencies on Idaho's Snake River
plains: ecological and management implications. Pages 4--10 \emph{in} E.
D. McArthur, E. M. Romney, S. Smith, and P. T. Tueller, editors.
Proceedings of the symposium on cheatgrass inva- sion, shrub die-off,
and other aspects of shrub biology and management. Forest Service
General Technical Report INT-276.

\hypertarget{ref-Wotton1993}{}
Wotton, B. M., and M. D. Flannigan. 1993. Length of the fire season in a
changing climate. The Forestry Chronicle 69:187--192.

\hypertarget{ref-Zhu2006}{}
Zhu, Z., J. Vogelmann, D. Ohlen, J. Kost, X. Chen, and B. Tolk. 2006.
Mapping existing vegetation composition and structure for the LANDFIRE
prototype project. Pages 197--215 \emph{in} Gen. tech. rep.
rmrs-gtr-175. fort collins: U.S. department of agriculture, forest
service, rocky mountain research station.


\end{document}
